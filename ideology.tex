\documentclass{article}
\usepackage[letterpaper,top=2cm,bottom=2cm,left=2cm,right=2cm,marginparwidth=1.75cm]{geometry}
\usepackage{graphicx} % Required for inserting images
\usepackage{amsfonts}
\usepackage{amssymb}
\usepackage{amsthm}
\usepackage{indentfirst}
\usepackage{amsmath}
\usepackage{mathrsfs}
\usepackage{xcolor}
\usepackage{tikz}
\usepackage[UTF8]{ctex}
\usetikzlibrary{positioning, shapes.geometric, graphs, quotes}
\linespread{1.5}

\usepackage{framed}
\definecolor{shadecolor}{RGB}{241, 241, 255}
\newcounter{problemname}
\newenvironment{problem}
{\begin{shaded}\stepcounter{problemname}\par\noindent\textbf{题目\arabic{problemname}. }}{\end{shaded}\par}
\newenvironment{solution}{\par\noindent\textbf{解答. }}{\par}
\newenvironment{Proof}{\par\noindent\textbf{证明. }}{$\blacksquare$\par}
\newenvironment{note}{\par\noindent\textbf{题目\arabic{problemname}的注记. }}{\par}
\newenvironment{define}{\par\noindent\textbf{定义. }}{\par}

\begin{document}
\bigskip
\begin{center}
\begin{tikzpicture}
    \node[draw] (start) {完备};
    \node[draw, below = of start] (step 1) {列紧};
    \node[draw, below = of step 1] (step 2) {有界};
    \node[draw, left = of step 2] (step 3) {完全有界};
    \node[draw, right = of step 1] (step 4) {自列紧};
    \node[draw, below = of step 4] (step 5) {有界闭集};
    \node[draw, right = of step 4] (step 6) {紧集};
    \graph{
        (step 1) -> (start);
        (step 3) -> (step 2);
        (step 4) -> (step 1);
    };
    \draw[->, >=latex] ([xshift = -1cm]step 1) -- ([xshift = -1cm]step 2) node[midway, left] {};
    \draw[->, >=latex] ([xshift = 1cm]step 2) -- ([xshift = 1cm]step 1) node[midway, left] {$\mathbb{R}^n$};
    \draw[->, >=latex] ([xshift = -1cm]step 4) -- ([xshift = -1cm]step 5) node[midway, left] {};
    \draw[->, >=latex] ([xshift = 1cm]step 5) -- ([xshift = 1cm]step 4) node[midway, left] {$\mathbb{R}^n$};
    \draw[->, >=latex] ([xshift = -1cm]step 1) -- ([xshift = -1cm]step 3) node[midway, left] {};
    \draw[->, >=latex] ([xshift = 1cm]step 3) -- ([xshift = 1cm]step 1) node[midway, left] {$\mathbb{R}^n$};
    \draw[->, >=latex] ([xshift = -1cm]step 4) -- ([xshift = -1cm]step 6) node[midway, left] {};
    \draw[->, >=latex] ([xshift = 1cm]step 6) -- ([xshift = 1cm]step 4) node[midway, below] {$\{\mathscr{X}, d\}$};
    \draw[->, >=latex] ([xshift = -1cm]step 6) -- ([xshift = -1cm]step 5) node[midway, left] {};
    \draw[->, >=latex] ([xshift = 1cm]step 5) -- ([xshift = 1cm]step 6) node[midway, left] {$\mathbb{R}^n$};
\end{tikzpicture}
\end{center}
\end{document}
